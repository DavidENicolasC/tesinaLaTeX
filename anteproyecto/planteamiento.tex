\section{Planteamiento del problema}

Los brazos robóticos se han introducido rápidamente en la industria, sin embargo, aún existen desafíos para posicionar de manera precisa el brazo robótico que dificultan su introducción en industrias donde el manejo de las materias primas es delicado, sobre todo cuando dicho posicionamiento tiene que realizarse de manera autónoma.
\newline\newline\newline
Ubicar las articulaciones de un brazo para llegar a una posición deseada es un problema clásico de la robótica llamado cinemática inversa. Existen métodos algebraicos, geométricos e iterativos para resolverlo, sin embargo, dependen de la cantidad de grados de libertad del robot, además de que consumen una gran cantidad de recursos computacionales, lo que dificulta su uso en un sistema crítico.