\subsection{Localización}

La posición en $(x, y, z)$ del sensor en la muñeca, será la posición que debe alcanzar el efector final del brazo robótico. Para obtener dicha posición, tomando el hombro como la posición $(0, 0, 0)$, se utilizaron las siguientes ecuaciones, considerando que aquellos ángulos con el símbolo $\theta^1$, son los ángulos medidos por el sensor colocado en el codo de la ortesis, y que aquellos con el símbolo $\theta^2$, son los ángulos medidos por el sensor colocado en el extremo del antebrazo:

\begin{equation}
	x = L_1 \cdot \cos \theta^1_x +  L_2 \cdot \sen \theta^2_x
	\label{eq:angulox}
\end{equation}

\begin{equation}
	y = L_1 \cdot \cos \theta^1_y +  L_2 \cdot \sen \theta^2_y
	\label{eq:angulox}
\end{equation}

\begin{equation}
	z = L_1 \cdot \cos \theta^1_z +  L_2 \cdot \sen \theta^2_z
	\label{eq:angulox}
\end{equation}

Donde $L_1$ es la longitud del brazo de la ortesis, desde el hombro (siendo la posición $(0, 0, 0)$) hasta el sensor colocado en el codo, y $L_2$ es la longitud del antebrazo de la ortesis, desde el sensor colocado en el codo, hasta el sensor colocado en el extremo del antebrazo.